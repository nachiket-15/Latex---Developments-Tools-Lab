\documentclass[conference]{IEEEtran}
\IEEEoverridecommandlockouts
% The preceding line is only needed to identify funding in the first footnote. If that is unneeded, please comment it out.
\usepackage{cite}
\usepackage{amsmath,amssymb,amsfonts}
\usepackage{algorithmic}
\usepackage{graphicx}
\usepackage{textcomp}
\usepackage{xcolor}
\def\BibTeX{{\rm B\kern-.05em{\sc i\kern-.025em b}\kern-.08em
    T\kern-.1667em\lower.7ex\hbox{E}\kern-.125emX}}
\begin{document}

\title{Identifying Software Engineering Challenges in
Software SMEs: A Case Study in Thailand\\
{\footnotesize \textsuperscript{*}Note: Sub-titles are not captured in Xplore and
should not be used}
\thanks{Identify applicable funding agency here. If none, delete this.}
}

\author{\IEEEauthorblockN{1\textsuperscript{st}Chaiyong Ragkhitwetsagul*
, Jens Krinke†
, Morakot Choetkiertikul*
, Thanwadee Sunetnanta*
, Federica Sarro†
*SERU}
\IEEEauthorblockA{\textit{ Faculty of Information and Communication Technology (ICT)} \\
\textit{Mahidol University}\\
Thailand
†CREST, UCL Computer Science, University College London, UK}
}

\maketitle

\begin{abstract}
Small and medium-sized software enterprises
(SSMEs) are a vital part of emerging markets. Due to their size,
they are not capable of adopting advanced software engineering
techniques or automated software engineering tools in the same
way large and ultra-large companies are. We study the software
engineering challenges in SSMEs in Thailand, an emerging in software development, using semi-structured interviews
with four SSMEs. After performing a thematic analysis of the
interview transcripts, we found a number of common challenges
such as lack of testing, code-related issues, and inaccurate effort
estimation. We observed that in order to introduce advanced software engineering tools and techniques, SSMEs need
to adopt contemporary best practices in software engineering like
automated testing, continuous integration and automated code
review. Moreover, we suggest that software engineering research
engage with SSMEs to enable them to improve their knowledge
and adopt more advanced software engineering practices.
Index Terms—empirical study, case study, software SMEs
\end{abstract}

\begin{IEEEkeywords}
component, formatting, style, styling, insert
\end{IEEEkeywords}

\section{Introduction}
Small and medium-sized enterprises (SMEs) are vital to
economic growth of a country and account for 95–99 percent of all enterprises in 36 OECD member countries . Software
SMEs (SSMEs) including software startups are among the
key drivers in software industry and important for a country’s
competitiveness and innovation . For example, a recent
study shows that the Thai software industry accounts for
4 billion in 2020 with a steady growth over the years. The
country’s software sector comprises of over 8,000 software
companies, large and small, with more than 100,000 software
engineers. With the transformation plan known as Thailand
4.0, the software industry has become a cornerstone for the
long-term plan of the country as it supports the development
of technology clusters and future industries [4], [5]. Similar
trends are also found in other Asian developing countries
Nonetheless, SSMEs are facing several challenges while
developing their software products. A technical report some of the major weaknesses of SSMEs include management, quality assurance, and project assessment
and control. Unlike large or ultra-large software companies,
SSMEs do not have much manpower. Moreover, SSMEs
may still use traditional software development approaches
and tools and rely heavily on manual tasks performed by
their programmers, which are costly and tedious approaches.
Thus, by identifying the software engineering challenges faced
by SSMEs, we can recommend state-of-the-art automated software engineering (ASE) techniques that SSMEs can adopt
to reduce the cost and increase the speed of their software
development and delivery. Only when the challenges leading
to the weaknesses are understood can they be addressed and
advanced software engineering practices be adopted.
Our study aims to fill in the gap by investigating the  engineering challenges that specifically occur in SSMEs
located in the Asian culture, especially in Thailand, and also to
study their tool usage for future recommendation of automated
software engineering (ASE) tools and techniques. Due to the
different working cultures from Western region [10], this study
reveals new insights into Thai, and possibly Asian, SSMEs.

\section{The study}
This study investigates current practices being used in the
day-to-day software development routines in Thai SSMEs and
identifies the challenges they are facing in terms of costs, time,
and software quality that prevent the adoption of advanced
software engineering practices. According to the guidelines
by Benbasat et al. [11], we aim to study a few entities (i.e.,
SSMEs) in their natural setting without experimental control or
manipulation. Thus, a case study is an appropriate method for
our aim. We use semi-structured interviews [12] and thematic
analysis [13] to extract the common challenges found by the
companies. We start by identifying research questions. Then,
we design the interview guide and perform semi-structured
interviews with the companies’ employees. Lastly, we perform
thematic analysis on the transcript and create a theme map to
identify common software engineering challenges.

\subsection{Research Questions}We aim to answer following research questions:
\begin{enumerate}
\item \textbf{RQ1 (SE Challenges)}: What are the challenges in the
SSMEs’ day-to-day software development? The answer
to this research question will identify the issues that need
to be addressed in order to adopt advanced software
engineering practices.

\item \textbf{RQ2 (Current Practices)}: What tools are being used?
By considering the tools the companies are using as a
proxy, we can gain insights into their current practices.
The answer to this question will help us to align  of advanced practices and automated tools
and techniques with the current practices
\end{enumerate}

\begin{table}[htpb]
\caption{Studied ssmes}
\begin{center}
\begin{tabular}{c | c | c}
\hline
Company & Products & No of developers\\[0.5ex]
\hline 
Company A & E Learning Platform & 15\\[0.5ex]
ProGaming & Web and Mobile games & 10\\[0.5ex]
Roots & Enterprise solutions & 40\\[0.5ex]
Zwiz & AI chatbot & 6\\[0.5ex]
\hline
\end{tabular}
\end{center}
\end{table}

\subsection{The Studied Companies}
We study the challenges and practices of four Thai SSMEs:
Company A (pseudonym), ProGaming, Roots, and Zwiz.AI.
An overview of the four companies is shown in Table I.
Their software products/services cover different business types
including games, e-learning, chatbots, and enterprise solutions.
Company A offers an online e-learning platform with more
than 1,800 courses, 10,000 users. ProGaming has developed more than 50 games, which
have been downloaded 2 million times. They also provide
game development services to Thai organizations. Roots offers
business consultancy and serves several large enterprises in
Thailand. Zwiz.AI provides AI chatbot services which are
serving more than 30,000 businesses and helping more than
10 million users. All of them are small (<50 employees) and
three of them are very small (<25 employees).

\subsection{Interview Design}\label{AA}
The goal for the interviews is to explore the challenges
in software development in the four companies. We perform
semi-structured interviews which are based on a general
grouping of topics and questions, instead of a predefined set
of questions [12]. This method performs a verbal discussion
with one programmer at a time. The interview script is updated
after each interview especially when the interviewees start
providing the same answers (saturation effect). The interview
process terminates when no new insights or observations are
made. The interview is recorded and then transcribed. The
maximum duration of each interview is set to 30 minutes.
The interview participants are suggested by the executives
of the companies based on their suitability for the project.
The subjects must satisfy the following inclusion criteria to be
included in the interview: (1) the subject must be a full-time
employee and must have worked at the company for at least
one month, (2) the subject must be involved in the software
development such as a CTO, a technical lead, or a developer.
To address the possibility of participants being forced by
their managers to participate, the authors took multiple steps
to mitigate the issues and the participation or non-participation
should not have had any negative effect on the employees.

\subsection{Semi Structured Interviews}
The semi-structured interviews occurred at the companies’
offices in February 20201
. The first author visited each
company and performed the semi-structured interviews in a dedicated room provided by the companies. The interviews
were in Thai and the recordings from the interviews were
transcribed. The initial questions used for the interviews are
listed in Table II. These questions were used to start the
conversations and the interviewer could ask more follow-up
questions depending on the answers of each interviewee.

\begin{table}[htpb]
\caption{Studied ssmes}

\begin{tabular}{c c}
\hline 
& Please explain your role at the company.\\
& How long have you been working at the company? \\[0.5ex]
\hline  
RQ1 & Please explain your day-to-day activity.\\
&How do you develop the software product?\\
&What works well in your company’s software development?\\
&What would you suggest to improve?\\[0.5ex]
\hline
RQ2 & Please explain the tools you use during software development\\
\hline
\end{tabular}

\end{table}
\subsection{Thematic Analysis}
After transcribing the interview recordings, reflexive  analysis, which is widely used to analyse qualitative
data, was applied on the transcript units. We followed the
guidelines by Braun and Clarke [13] with a deductive approach
in which coding and theme development were directed by the
challenges in software engineering. The first, third, and fourth
authors (the coders) were assigned a set of transcripts to code.
They carefully went through a transcript and assigned one or
more codes to the transcript units. The codes were in English
to facilitate generalisation and enable working with authors
from different countries and comprised of software challenges such as “lack of unit testing”, “coding style
violations”, or “mismatched requirements”. The coders shared
the set of codes and kept adding new codes if the existing ones
did not match with what they found in the transcript. Then,
the assigned codes by one coder were validated by another
coder and each conflict was discussed until a consensus was
found. Next, the first and second authors identified themes
by sorting the validated codes into groups according to their
content, splitting codes into different groups, or discarding the
codes. The process was repeated multiple time until no further
change could be made. Finally, a set of common challenges
in the companies’ software development emerged.

\section{Results}

We interviewed 20 software engineers and software-related
employees (details are shown in Table III). The interviewees’
roles range from management positions , Officers (CTO) to developers. The working experience
ranges from 4 months to 11 years.

\subsection{RQ1: Identified SE Challenges}The identified software engineering challenges are shown in
Figure 1. We discuss each of the challenges in detail below
The identified software engineering challenges are shown in
Figure 1. We discuss each of the challenges in detail below
\begin{table}[htpb]
\caption{Studied ssmes}

\begin{tabular}{c c}
\hline 
Company & Participant(Experience at the company in years)\\
\hline  
Company A & Chief Technology Officer \\[0.5ex]
\hline
RQ2 & Please explain the tools you use during software development\\
\hline
\end{tabular}
\end{table}

\begin{enumerate}
\item Lack of testing:
The most prominent challenge that
occurred during the interviews, coding, and theme mapping
is the lack of software testing, which occurs across all four
companies. This includes the lack of using automated testing
(mentioned 4 times in the interviews by different, lack of unit testing (mentioned 16 times), low test
quality (mentioned 6 times), and lack of testing knowledge
and time for testing (mentioned 3 times). The companies focus
mainly on implementing the software and testing is mostly
performed manually at the end of the development. Unit
testing is not usually performed and the developers also do
not value unit tests during the development. One interviewee
also mentioned a lack of regression testing. This is in line
with the previous findings that tests are not well adopted in
SSMEs [14] and in Thailand [10]. Moreover, the adoption of
automated testing is a challenge because of the lack of time
and knowledge to study and deploy in the companies

\item Code-related issues:
The second challenge is about
code-related issues. This challenge also applies to all the four
companies. We found three issues as follow: (1) the lack of
code analysis and measurements (mentioned 4 times) such as
not using any static or dynamic analysis tools to analyse source
code, (2) low code quality (mentioned 6 times) including
buggy code, duplicated code, legacy code that is hard to
understand, and code that is difficult to reuse and needs to be
refactored, and (3) difficulty of maintaining or using coding
conventions in the company (mentioned 4 times).

\item Unclear and incomplete requirements:
The third challenge is about requirements (mentioned 13 times by 3 companies). The interviewees explained that they found some of
the gathered requirements unclear, too vague, or incomplete.
Some of the requirements mismatched the customers’ needs.
Moreover, they also mentioned an issue of “scope creep” due
to new requirements being added during the development.

\item Inaccurate effort estimation: 
The fourth challenge is
difficulties in estimating the amount of work to be done (mentioned 11 times by 3 companies). The interviewees explained
several issues. For example, they often needed to put in more
time than what was planned for and worked overtime. There
were also unplanned tasks added in a sprint. Moreover, some
of the interviewees mentioned that they could not make correct
estimation to complete the assigned tasks
\end{enumerate}


\subsection{RQ2: Identified Current Practices}
To gain more insights into the current practices of the
companies, we asked the participants about automated tools
that they regularly use. We observed some interesting findings
as follows.\\
\begin{enumerate}

  \item Unit testing frameworks: Despite using continuous integration, the companies did not use unit testing frameworks.
Interestingly, there was one interviewee mentioning using
JEST and Mocha, the unit testing frameworks for Javascript.
However, the other interviewees from the same company
mentioned the lack of unit testing adoption. This mentioning of
unit testing frameworks possibly comes from the interviewee’s
own expertise and may indicate that the company still needs
to improve the company-wide adoption of testing

  \item Code structure and formatting are locally enforced: The
interviewees from the four companies explained that they used
linters, such as ESLint or PyLint, and code formatters, such
as Prettier, in their IDEs. Nonetheless, from answers to RQ1,
we found that some companies do not perform code review
nor include any automated tools during code review. Thus,
code structure, formatting, and coding conventions may not
be enforced at company level but at individual level.
\item ) Mixed use of project management tools: We observed
that the four companies use dedicated project management
tools such as Asana, Trello, or Clickup, to manage their tasks
and track their projects’ progresses. However, two companies
also mentioned using Google Sheets, the general purpose
spreadsheet tool to handle some parts of their projects.
\item Mixed use of communication tools: The interviewees
from three companies mentioned that they mainly used LINE
and Discord, which are popular chat applications in Thailand.
Only one company used a dedicated communication tool,
MatterMost. Since LINE and Discord are mainly personal
chat applications, this shows that the developers may mix
their work with their personal life. In part, this can be
from the specific culture in Thailand that people tend to respond faster on their personal chat platforms than workrelated platforms. There is no mention of having internal QA
forums or knowledge management platforms possibly due to
the relatively small size of the companies.


\end{enumerate}



\section{Discussion}
Outcomes: The case study is part of an ongoing multinational industry-academia collaboration project to support
SSMEs in the adoption of suitable automated software engineering tools and techniques, similar to those that are
successfully used in large and ultra-large companies.
The presented study was intended to identify the challenges
of SSMEs in an emerging market like Thailand and address the
identified challenges through automated software engineering.
While performing this study as part of the project, we identified one particular challenge which needs to be addressed
before further advanced practices or automated software engineering can be adopted: lack of testing. Contemporary best
practices in software engineering are to use automated testing
tools to test the software in continuous integration settings, and
automated software engineering often requires a large amount
of testing. Without automated testing, there is no foundation
for improving the software development process.
Lessons Learned: An important lesson learned is the
observation that before one can attempt to support SSMEs
in the adoption of automated software engineering tools and
techniques, one must first ensure that the SSMEs have adopted
contemporary best practices in software engineering. In the
context of the ongoing project, the study results led to a
change of the project’s focus and aims: Instead of introducing
the most advanced automated software engineering tools and
techniques, the project will focus on the effect of the adoption
of best practices in software development like automated
testing, continuous integration, and automated code review.
Takeaway Messages: Knowledge transfer between software engineering research and industry has been a key factor in
the success of the adoption of automated software engineering
tools and techniques. Such cooperation and knowledge transfer
usually occurs in large and ultra-large companies. SSMEs, especially in emerging markets, often are not able to participate
in such cooperation and knowledge transfer. Research should,
therefore, engage SSMEs in order to facilitate the adoption of
automated software engineering tools and techniques so that
the gap between SSMEs and large and ultra-large companies
does not widen.

\section{Related Work}
Laporte et al. [15] and Habra et al. [16] report their experience of applying ISO29110, a software process guidelines
for very small entities (VSEs), to companies. Basri et al. [17]
study acceptance of software process improvements in VSEs.
Sunetnunta et al. [9] report preliminary results from a gap
analysis of 39 Thai VSEs’ software development activities.
They found that the major weaknesses include configuration
management, quality assurance, and project assessment and
control. Phongpaibul and Boehm [10] show that the cultural
factors affect the adoption of software process improvement
in Thailand. Tuape and Ayalew [18] identify that technical
factors are the key factors affecting software processes of
African SSMEs. Larrucea et al. [2] also report common
barriers to software process improvement in SSMEs. Our study
investigates the challenges that are not only related to the
software process but also to best practices and tools. We
found some results that correlate with these previous studies
as discussed in Section III.
A few studied identify important factors or best
practices in software development such as unit testing, code
reviews, coding guidelines, code analysis, version control, and
project management tools. Garousi et al. discuss the
challenges of test automation management. Ponsard and Deprez report that requirements, technical debt, test/release
management, and project management are challenges faced
by Belgian SSMEs. Jones et al. study the challenges
of DevOps adoption in UK SSMEs. Compared to our study,
we found similar challenges such as lack of unit testing and
requirement management.
Other studies identify software engineering challenges based
on different phases throughout the software development life
cycle. Shah et al. [24] identify challenges in the requirement engineering such as communication gap and customer
involvement. Mantyl ¨ a et al. and Shanin et al. identify ¨
challenges in software delivery, deployment, and CI/CD.

\section{Threats To Validity}
\textbf{External validity:} The study is performed on four Thai
SSMEs and the findings may not be generalised to all SSMEs
in Thailand and other developing countries. We mitigated this
threat by selecting SSMEs from different domains. Due to
cultural differences, some of the findings may only be found
in Asian regions and not be applied to SSMEs in other regions.\\
\textbf{Internal validity:} The selection of participants may affect
the validity of the identified challenges. We mitigated this
threat by using an inclusion criteria to control the experience
with the company’s software development process.

\section{Conclusions }
This paper studies the software engineering challenges that
SSMEs experience and the current practices used in the
companies. The study is conducted as a case study using semistructured interviews and thematic analysis of four Thai software companies. The result shows that the studied companies
are facing challenges in contemporary software best practices,
which are often well-adopted in large or ultra-large software
companies, such as software testing, code-related issues, and
requirements. Moreover, we found that the SSMEs are using
general-purpose tools such as spreadsheets or personal chat
applications for project management and communication instead of dedicated tools. The lack of testing not only affects
the quality of the software, but it also prevents the adoption of
advanced practices and automated software engineering tools.
Such practices and tools are often the result of cooperation
and knowledge transfer between software engineering research
and large or ultra-large software companies. We call for more
engagement of SSMEs in software engineering research to
ensure that no one is left behind.

\section*{Acknowledgements}
This work is supported by the Newton Fund and funded by
the Royal Academy of Engineering, UK (Industry Academia
Partnership Programme – 18/19 – IAPP18-19/74). It was
approved by both universities’ research ethics boards. The
authors have no financial or proprietary interests in the four
studied companies.


\begin{thebibliography}{00}
\bibitem{b1} OECD, OECD SME and Entrepreneurship Outlook 2005. OECD
Publishing, 2005.
\bibitem{b2} X. Larrucea, R. V. O’Connor, R. Colomo-Palacios, and C. Y. Laporte,
“Software Process Improvement in Very Small Organizations,” IEEE
Software, vol. 33, no. 2, pp. 85–89, Mar 2016.
\bibitem{b3} Digital Economy Promotion Agency, “Thailand’s Digital Industry Survey 2020 and 3-Year Prediction,” https://www.depa.or.th/th/article-view/
index-3-years-since-2563, 2021
\bibitem{b4} B. D. Royal Thai Embassy, “Thailand 4.0 - Agenda 2: Development
of Technology Cluster and Future Industries,” https://thaiembdc.org/
thailand-4-0-2, 2016
\bibitem{b5} R. Nicole, ``Title of paper with only first word capitalized,'' J. Name Stand. Abbrev., in press.
\bibitem{b6} Y. Yorozu, M. Hirano, K. Oka, and Y. Tagawa, ``Electron spectroscopy studies on magneto-optical media and plastic substrate interface,'' IEEE Transl. J. Magn. Japan, vol. 2, pp. 740--741, August 1987 [Digests 9th Annual Conf. Magnetics Japan, p. 301, 1982].
\bibitem{b7} M. Young, The Technical Writer's Handbook. Mill Valley, CA: University Science, 1989.
\end{thebibliography}
\vspace{12pt}


\end{document}
